\documentclass{article}
\usepackage[utf8]{inputenc}
\usepackage{graphicx}
\usepackage{mathtools}

\title{Data Structure: Theoretical Approach}
\author{ Durgesh Raghuvanshi\\
         B-Tech Department of Computer Science,\\
IILM Academy of Higher Learning, Greater Noida, Uttar Pradesh, India}


\begin{document}

\section{ABSTRUCT}
Run with accordance with significance. The first if
these this paper explains about the basic terminologies
used in this paper in data structure. Better running
times will be other constraints, such as memory use
which will be paramount. The most appropriate data
structures and algorithms rather than through hacking
removing a few statements by some clever coding.
Data structures serve as the basis for abstract data
types (ADT). "The ADT defines the logical form of
the data type. The data structure implements the
physical form of the data type."Different types of data
structures are suited to different kinds of applications,
and some are highly specialized to specific tasks. For
example, relational databases commonly use B-tree
indexes for data retrieval, while compiler
implementations usually use hash tables to look up
identifiers
\section{INTRODUCTION}
Data structures serve as the basis for abstract data
types (ADT). "The ADT defines the logical form of
the data type. The data structure implements the
physical form of the data type."Different types of data
structures are suited to different kinds of applications,
and some are highly specialized to specific tasks. For
example, relational databases commonly use B-tree
indexes for data retrieval, while compiler
implementations usually use hash tables to look up
identifiers. Data structures provide a means to manage
large amounts of data efficiently for uses such as large
databases and internet indexing services. Usually,
efficient data structures are key to designing efficient
algorithms. Some formal design methods and
programming languages emphasize data structures,
rather than algorithms, as the key organizing factor in
software design. Data structures can be used to
organize the storage and retrieval of information
stored in both main memory and secondary memory.
Data structures are generally based on the ability of acomputer to fetch and store data at any place in its
memory, specified by a pointer—a bit string,
representing a memory address, that can be itself
stored in memory and manipulated by the program.
Thus, the array and record data structures are based on
computing the addresses of data items with arithmetic
operations, while the linked data structures are based
on storing addresses of data items within the structure
itself. Many data structures use both principles,
sometimes combined in non-trivial ways (as in XOR
linking).[citation needed] \\
The implementation of a data structure usually
requires writing a set of procedures that create and
manipulate instances of that structure. The efficiency
of a data structure cannot be analyzed separately from
those operations. This observation motivates the
theoretical concept of an abstract data type, a data
structure that is defined indirectly by the operations
that may be performed on it, and the mathematical
properties of those operations (including their space
and time cost).[citation needed]An array is a number
of elements in a specific order, typically all of the
same type (depending on the language, individual
elements may either all be forced to be the same type,
or may be of almost any type). Elements are accessed
using an integer index to specify which element is
required. Typical implementations allocate contiguous
memory words for the elements of arrays (but this is
not necessity). Arrays may be fixed-length or
resizable. A linked list (also just called list) is a linear
collection of data elements of any type, called nodes,
where each node has itself a value, and points to the
next node in the linked list. The principal advantage
of a linked list over an array, is that values can always
be efficiently inserted and removed without relocating
the rest of the list. Certain other operations, such as
random access to a certain element, are however
slower on lists than on arrays. Most assembly languages and some low-level languages, such as
BCPL (Basic Combined Programming Language),
lack built-in support for data structures. On the other
hand, many high-level programming languages and
some higher-level assembly languages, such as
MASM, have special syntax or other built-in support
for certain data structures, such as records and arrays.
\section{Sequential search}
When data items are stored in a collection such as a
list, we say that they have a linear or sequential
relationship. Each data item is stored in a position
relative to the others. In Python lists, these relative
positions are the index values of the individual items.
Since these index values are ordered, it is possible for
us to visit them in sequence. This process gives rise to
our first searching technique, the sequential search.
Starting at the first item in the list, we simply move
from item to item, following the underlying sequential
ordering until we either find what we are looking for
or run out of items. If we run out of items, we have
discovered that the item we were searching for was
not present. 
\section{Algorithm complexity}
\begin{center}
\begin{tabular}{|c|c|c|}
      
    \hline
     Algorithm &Best case    &  Expected\\
     \hline
      Selection sort   & O(N2) & O(N2)\\
      \hline
      Merge sort& O(NlogN)& O(NlogN)\\
      \hline
      Linear search & O(1)& O(N)\\
      \hline
      Binary search& O(1) & O(logN)\\
      \hline  
    \end{tabular}
\end{center}
\section{Depth of node}
The depth of node is the length of the path from the
root to the node. A rooted tree with only one node has
a depth of zero.
\section{Threaded binary tree }

In a threaded binary tree all the null pinters which
wasted the space in linked representation is converted
into useful links called threads thus representation of a
binary tree using these threads is called threaded
binary tree. 
\section{Analysis of sequential search}
To analyze searching algorithms, we need to decide
on a basic unit of computation. Recall that this is
typically the common step that must be repeated in
order to solve the problem. For searching, it makes
sense to count the number of comparisons performed.
Each comparison may or may not discover the item
we are looking for. In addition, we make another
assumption here. The list of items is not ordered in
any way. The items have been placed randomly into
the list. In other words, the probability that the item
we are looking for is in any particular position is
exactly the same for each position of the list.
If the item is not in the list, the only way to know it is
to compare it against every item present. If there are
\(n\) items, then the sequential search requires \(n\)
comparisons to discover that the item is not there. In
the case where the item is in the list, the analysis is
not so straightforward. There are actually three
different scenarios that can occur. In the best case we
will find the item in the first place we look, at the
beginning of the list. We will need only one
comparison. In the worst case, we will not discover
the item until the very last comparison, the nth
comparison.
\section{Binary search}
Binary search is a fast search algorithm with run-time
complexity of Ο(log n). This search algorithm works
on the principle of divide and conquer. For this
algorithm to work properly, the data collection should
be in the sorted form.
Binary search looks for a particular item by
comparing the middle most item of the collection. If a
match occurs, then the index of item is returned. If the
middle item is greater than the item, then the item is
searched in the sub-array to the left of the middle
item. Otherwise, the item is searched for in the subarray to the right of the middle item. This process
continues on the sub-array as well until the size of the
sub array reduces to zero. B-trees are generalizations
of binary search trees in that they can have a variable
number of sub trees at each node. While child-nodes
have a pre-defined range, they will not necessarily be
filled with data, meaning B-trees can potentially waste
some space. The advantage is that B-trees do not need
to be re-balanced as frequently as other self-balancing
trees.
Due to the variable range of their node length, B-trees
are optimized for systems that read large blocks of
data. They are also commonly used in databases. A
ternary search tree is a type of tree that can have 3
nodes: a lo kid, an equal kid, and a hi kid. Each node
stores a single character and the tree itself is ordered
the same way a binary search tree is, with the
exception of a possible third node. Searching a ternary
search tree involves passing in a string to test whether
any path contains it. The time complexity for
searching a balanced ternary search tree is O(log n).
\begin{figure}
    \centering
    \includegraphics[width=\linewidth]{fig.png}
    \caption{Flowchart}
    \label{fig:my_label}
\end{figure}
Binary Search Tree, is a node-based binary tree
data structure which has the following properties:
The left subtree of a node contains only nodes with
keys lesser than the node’s key. The right subtree of a
node contains only nodes with keys greater than the
node’s key. The left and right subtree each must also
be a binary search tree. There must be no duplicate
nodes. delete operation is discussed. When we delete
a node, three possibilities arise. 1) Node to be deleted
is leaf: Simply remove from the tree.3) Node to be
deleted has two children: Find in order successor of
the node. Copy contents of the in order successor to
the node and delete the in order successor. Note that
in order predecessor can also be used.2) Node to be
deleted has only one child: Copy the child to the node
and delete the child The important thing to note is, in
order successor is needed only when right child is not
empty. In this particular case, in order successor can
be obtained by finding the minimum value in right
child of the node. Time Complexity: The worst case
time complexity of delete operation is O(h) where h is
height of Binary Search Tree. In worst case, we may
have to travel from root to the deepest leaf Now when
we want to search for an item, we simply use the hash
function to compute the slot name for the item and
then check the hash table to see if it is present. This
searching operation is \(O(1)\), since a constant
amount of time is required to compute the hash value
and then index the hash table at that location. If
everything is where it should be, we have found a
constant time search algorithm.
You can probably already see that this technique is
going to work only if each item maps to a unique
location in the hash table. For example, if the item 44
had been the next item in our collection, it would have
a hash value of 0 (\(44 \% 11 == 0\)). Since 77 also
had a hash value of 0, we would have a problem.
According to the hash function, two or more items
would need to be in the same slot. This is referred to
as a collision (it may also be called a “clash”).
Clearly, collisions create a problem for the hashing
technique. We will discuss them in detail later. We
now return to the problem of collisions. When two
items hash to the same slot, we must have a
systematic method for placing the second item in the
hash table. This process is called collision resolution.
As we stated earlier, if the hash function is perfect,
collisions will never occur. However, since this is
often not possible, collision resolution becomes a very
important part of hashing. 


\section{Binary search tree}
It is observed that BST's worst-case performance is 
closest to linear search algorithms, that is Ο(n). In 
real-time data, we cannot predict data pattern and 
their frequencies. So, a need arises to balance out the 
existing BST. 
Named after their inventor Adelson, Velski& Landis, 
AVL trees are height balancing binary search tree. 
AVL tree checks the height of the left and the right 
sub-trees and assures that the difference is not more 
than 1. This difference is called the Balance Factor. 
Here we see that the first tree is balanced and the next 
two trees are not balanced −In the second tree, the left 
subtree of C has height 2 and the right subtree has 
height 0, so the difference is 2. In the third tree, the 
right subtree of A has height 2 and the left is missing, 
so it is 0, and the difference is 2 again. AVL tree 
permits difference (balance factor) to be only 1.If the 
difference in the height of left and right sub-trees is 
more than 1, the tree is balanced using some rotation 
techniques. In our example, node A has become 
unbalanced as a node is inserted in the right subtree of 
A's right subtree. We perform the left rotation by 
making A the left-subtree of B Right Rotation AVL tree may become unbalanced, if 
a node is inserted in the left subtree of the left subtree. 
The tree then needs a right rotation.AVL Rotations To 
balance itself, an AVL tree may perform the 
following four kinds of rotations − 
Left rotation Right rotation Left-Right rotation RightLeft rotation The first two rotations are single 
rotations and the next two rotations are double 
rotations. To have an unbalanced tree, we at least 
need a tree of height 2. With this simple tree, let's 
understand them one by one. 
Left Rotation If a tree becomes unbalanced, when a 
node is inserted into the right subtree of the right 
subtree, then we perform a single left rotation −RightLeft Rotation The second type of double rotation is 
Right-Left Rotation. It is a combination of right 
rotation followed by left rotation. As depicted, the 
unbalanced node becomes the right child of its left 
child by performing a right rotation. 
Left-Right Rotation Double rotations are slightly 
complex version of already explained versions of 
rotations. To understand them better, we should take 
note of each action performed while rotation. Let's 
first check how to perform Left-Right rotation. A leftright rotation is a combination of left rotation 
followed by right rotation.
An internal sort is any data sorting process that takes 
place entirely within the main memory of a computer. 
This is possible whenever the data to be sorted is 
small enough to all be held in the main memory. For 
sorting larger datasets, it may be necessary to hold 
only a chunk of data in memory at a time, since it 
won’t all fit. The rest of the data is normally held on 
some larger, but slower medium, like a hard-disk. 
Any reading or writing of data to and from this slower 
media can slow the sortation process considerably. 
This issue has implications for different sort 
algorithms.
\section{Some common internal sorting algorithms include:}
Bubble Sort Insertion Sort Quick Sort Heap Sort 
Radix Sort Selection sort Consider a Bubblesort, 
where adjacent records are swapped in order to get 
them into the right order, so that records appear to 
“bubble” up and down through the dataspace. If this 
has to be done in chunks, then when we have sorted 
all the records in chunk 1, we move on to chunk 2, but 
we find that some of the records in chunk 1 need to “bubble through” chunk 2, and vice versa (i.e., there 
are records in chunk 2 that belong in chunk 1, and 
records in chunk 1 that belong in chunk 2 or later 
chunks). This will cause the chunks to be read and 
written back to disk many times as records cross over 
the boundaries between them, resulting in a 
considerable degradation of performance. If the data 
can all be held in memory as one large chunk, then 
this performance hit is avoided. On the other hand, 
some algorithms handle external sorting rather better. 
A Merge sort breaks the data up into chunks, sorts the 
chunks by some other algorithm (maybe bubblesort or 
Quick sort) and then recombines the chunks two by 
two so that each recombined chunk is in order. This 
approach minimises the number or reads and writes of 
data-chunks from disk, and is a popular external sort 
method. It is useful to understand how storage is 
managed in different programming languages and for 
different kinds of data. Three important cases are: 
static storage allocation stack-based storage allocation 
heap-based storage allocation Static Storage 
Allocation Static storage allocation is appropriate 
when the storage requirements are known at compile 
time. For a compiled, linked language, the compiler 
can include the specific memory address for the 
variable or constant in the code it generates. (This 
may be adjusted by an offset at link time.) Examples: 
code in languages without dynamic compilation all 
variables in FORTRAN IV global variables in C, Ada, 
Algol constants in C, Ada, Algol Stack-Based Storage 
Allocation Stack-based storage allocation is 
appropriate when the storage requirements are not 
known at compile time, but the requests obey a lastin, first-out discipline. Examples: local variables in a 
procedure in C/C++, Ada, Algol, or Pascal procedure 
call information (return address etc).Stack-based 
allocation is normally used in C/C++, Ada, Algol, and 
Pascal for local variables in a procedure and for 
procedure call information. It allows for recursive procedures, and also allocates data only when the 
procedure or function has been called -- but is 
reasonably efficient at the same time. Typically a 
pointer to the base of the current stack frame is held in 
a register, say R0. A reference to a local scalar 
variable can be compiled as a load of the contents of 
R0 plus a fixed offset. Note that this relies on the data 
having known size. To compile a reference to a 
dynamically sized object, e.g. an array, use 
indirection. The stack contains an array descriptor, of 
fixed size, at a known offset from the base of the 
current stack frame. The descriptor then contains the 
actual address of the array, in addition to bounds 
information. References to non-local variables can be 
handled by several techniques -- the most common is 
using static links. This is beyond the scope of what 
we'll cover in 341 this year. Most variable references 
are either to local variables or global variables, so 
often compilers will handle global variable references 
more efficiently than references to arbitrary non-local 
variables. Scalar local variables (especially 
parameters) can be handled efficiently as they are 
often passed through registers. 
There are two important limitations to pure stackbased storage allocation.
First, the only way to return data from a procedure or 
function is to copy it -- if you try to simply return a 
reference to it, the storage for the date will have 
vanished after the procedure or function returns. This 
isn't an issue for scalar data (integers, floats, 
booleans), but is an issue for large objects such as 
arrays. For this reason you can't for example return a 
loclly declared array from a function in C. Second, 
you can't return a procedure or function as a value, or 
assign a procedure or function to a global variable 
(procedures or function values aren't first class 
citizens)

\section{Comparison between linear search and binary search}
\begin{center}  
\begin{tabular}{|l|c|r|}  
\hline 
Basis for comparison& Linear search& Binary search\\
\hline
Time complexity O(N)&O(N)&O(0LOG2N)\\
\hline

Best case time & First element 0(1) & Centre element 0(1)\\
\hline
Prerequisite for an array &  Not required & Array must be sorted in order\\
\hline
Algorithm type & Iterative in nature & Divide and conquer in nature\\
\hline



\end{tabular}
\end{center}
\section{Conclusion }
This paper covered the basics of data structures. With 
this we have only scratched the surface.
Although we have built a good foundation to move 
ahead. Data Structures is not just limited to Stack, 
Queues, and Linked Lists but is quite a vast area.
There are many more data structures which include 
Maps, Hash Tables, Graphs, Trees, etc. Each data 
structure has its own advantages and disadvantages 
and must be used according to the needs of the 
application. A computer science student at least know 
the basic data structures along with the operations 
associated with them. Many high level and object 
oriented programming languages like C#, Java, 
Python come built in with many of these data 
structures. Therefore, it is important to know how 
things work under the hood. Dynamic data structures 
require dynamic storage allocation and reclamation. 
This may be accomplished by the programmer or may 
be done implicitly by a high-level language. It is 
important to understand the fundamentals of storage 
management because these techniques have
significant impact on the behavior of programs. The
basic idea is to keep a pool of memory elements that 
may be used to store components of dynamic data 
structures when needed. Allocated storage may be 
returned to the pool when no longer needed. In this 
way, it may be used and reused. This contrasts sharply 
with static allocation, in which storage is dedicated
or the use of static data structures. It cannot then be 
reclaimed for other uses, even when no needed for the 
static data structure. As a result, dynamic allocation 
makes it possible to solve larger problems that might 
otherwise be storage-limited. Garbage collection and 
reference counters are two basic techniques for 
implementing storage management. Combinations of 
these techniques may also be designed. Explicit 
programmer control is also possible. Potential pitfalls
of these techniques are garbage generation, dangling 
references, and fragmentation. High-level language 
may take most of the burden for storage management 
from the programmer. The concept of pointers or 
pointer variables underlies the use of these facilities, 
and complex alg.

\section{References}

1. Book of Data structures through C G. S Baluja.\\
2. Pieren Garry Department of computer science 
New York University.\\
3. Paul Xavier department of algorithms in c 
Amsterdam.\\
4. Surendrakumar Ahuja IItdelhi department of 
computer science delhi .\\
5. Nick jones department of data mining Australia.\\
6. Wikipedia sequential search.\\


  
      



\section{FOR EXAMPLE}\\


math equation $$y=x^2
\section{picture}\\
\begin{figure}
    \centering
    \includegraphics[width=\linewidth]{pic.png}
    \caption{random pic}
    \label{fig:my_label}
\end{figure}
   
   

      
    


\end{document}


    
    
    


